\documentclass[10pt]{beamer}
\usetheme{Boadilla}% theme général du diaporam
% paquets pour le français
\usepackage[T1]{fontenc}
\usepackage[utf8]{inputenc}
\usepackage{pgf}
\usepackage{color}
\usepackage{lmodern}
\usepackage{tikz}
\usepackage{eurosym}
\usepackage{verbatim}
\usepackage{multicol}
\usepackage[eulergreek,EULERGREEK]{sansmath}
\definecolor{nemo}{rgb}{1.,0.27,0.} 
\setbeamercolor{structure}{fg=nemo}
\setbeamercolor{alerted text}{fg=blue}
\usepackage{array}

\setbeamertemplate{footline}[text line]
{
  \insertshortauthor\\ \insertshortdate \hfill  \small{\insertpagenumber}
}

\makeatletter
\newcommand{\boldarrayrulewidth}{1\p@} 
\def\bhline{\noalign{\ifnum0=`}\fi\hrule \@height  
 \boldarrayrulewidth \futurelet \@tempa\@xhline}

\def\@xhline{\ifx\@tempa\hline\vskip \doublerulesep\fi
 \ifnum0=`{\fi}}
\newcommand{\ms}{\noalign{\vspace{3\p@ plus2\p@ minus1\p@}}}

\makeatother
\newcommand{\br}{\ms\bhline\ms}
\newcommand{\mr}{\ms\hline\ms}

\begin{document}

\title{\textbf{Helium Binding Energy in Helium-Vacancy Clusters}} 
\author{Sophie Blondel \& Crystal Jernigan}  
\date{\today}

\frame{\titlepage}

\begin{frame}{Formation Energy Data}
	Formation energy files only for V~$= 1, 2, 6, 14, 18, 19, 27, 32, 44$. \newline
	
	Example:
	
	\texttt{\#V \#He Ef \\
	1 ~0  ~~3.82644  \\                                                                                           
	1 ~1  ~~5.14166  \\                                                                                        
	1 ~2  ~~8.20919  \\                                                                                         
	1 ~3  ~~11.5304} \newline
	
	Binding energy formula:
	$$\text{E}_{\text{b}}(\text{He}_{\text{X}}, \text{V}_{\text{Y}}) =
	\text{E}_{\text{f}}(\text{He}_{\text{X}-1}, \text{V}_{\text{Y}}) +
	\text{E}_{\text{f}}(\text{He}_1, 0) -
	\text{E}_{\text{f}}(\text{He}_{\text{X}}, \text{V}_{\text{Y}})$$
\end{frame}

\begin{frame}{Binding Energy Data}
	Binding energy file for V up to $50$. \newline
	
	Example:
	
	\texttt{\#He \#V \#I E\_He ~~~~E\_V ~~~~~E\_I ~~~~~E\_t ~~~~~E\_mig
	~~~D\_0
	\\
	1 ~~0 ~0 ~Infinity Infinity Infinity 8.270E+0 1.300E-1 2.950E+10 \\
	2 ~~0 ~0 ~8.640E-1 Infinity Infinity 6.120E+0 2.000E-1 3.240E+10 \\
	3 ~~0 ~0 ~1.210E+0 Infinity Infinity 4.440E+0 2.500E-1 2.260E+10 \\
	4 ~~0 ~0 ~1.560E+0 Infinity Infinity 3.180E+0 2.000E-1 1.680E+10 } \newline
	
	Here we are only interested in the helium binding energy.
\end{frame}

\begin{frame}{Helium Formation Energy}
	$\text{E}_{\text{f}}(\text{He}_1, 0)$ is missing from our data:
    \begin{itemize}
      	\item[$\blacktriangleright$] can be computed with the formula and the
      	data we have.
    \end{itemize}
	$$\text{E}_{\text{b}}(\text{He}_{\text{X}}, \text{V}_{\text{Y}}) =
	\text{E}_{\text{f}}(\text{He}_{\text{X}-1}, \text{V}_{\text{Y}}) +
	\text{E}_{\text{f}}(\text{He}_1, 0) -
	\text{E}_{\text{f}}(\text{He}_{\text{X}}, \text{V}_{\text{Y}})$$
	
    \begin{figure}
        \includegraphics[width=0.7\textwidth]{heliumFormation}
    \end{figure}
	
	$\text{E}_{\text{f}}(\text{He}_1, 0) = 6.15$~eV will be used now.
\end{frame}

\begin{frame}{Rebuilding Binding Energies}
	Using only the given formation energies, one can now compute new binding
	energies:
	
    \begin{figure}
        \includegraphics[width=0.9\textwidth]{bindingEnergiesComparison}
    \end{figure}
\end{frame}


\begin{frame}{Formation Energy Data}
  	\begin{columns}[onlytextwith]
    	\begin{column}{0.5\textwidth}
      		\begin{figure}
        		\includegraphics[width=\textwidth]{formationEnergyBrian}
      		\end{figure}
    	\end{column}  
    	\begin{column}{0.5\textwidth}
      		\begin{figure}
        		\includegraphics[width=0.9\textwidth]{formationEnergySophie}
      		\end{figure}
    	\end{column}
  	\end{columns}
\end{frame}

\begin{frame}{From Juslin's Presentation}
	\begin{figure}
        \includegraphics[width=0.9\textwidth]{fitsBrian}
    \end{figure}
\end{frame}


\begin{frame}{Fitting Methods}
	\large
    \begin{itemize}
      	\item[$\blacktriangleright$] Piecewise $2$D polynomials fit with a
      	separation arround He/V~$=1$ where the order of the lower and higher
      	parts can be chosen independently \newline
      	\item[$\blacktriangleright$] Two-step $1$D polynomials fits:
    	\begin{itemize}
      		\item[-] Use only polynomials
      		\item[-] Piecewise fit first on the formation
      		energies as a function of He/V with a separation arround He/V~$=1$
      		\item[-] Fit then the parameters of those fits as a
      		function of V
      		\item[-] Each fit orders ($4$ in total) and the
      		separation can be changed
    	\end{itemize}
    \end{itemize}
\end{frame}

\begin{frame}{Example of Two-Step fit:}
	Order $1$ and order $3$ polynomials for V~$= 1$ 
	\begin{figure}
        \includegraphics[width=0.95\textwidth]{fit1DS1}
    \end{figure}
\end{frame}

\begin{frame}{Example of Two-Step fit:}
	Order $1$ and order $3$ polynomials for V~$= 2$
	\begin{figure}
        \includegraphics[width=0.95\textwidth]{fit1DS2}
    \end{figure}
\end{frame}

\begin{frame}{Example of Two-Step fit:}
	Order $1$ and order $3$ polynomials for all V
	\begin{figure}
        \includegraphics[width=0.95\textwidth]{fit1DS3}
    \end{figure}
\end{frame}

\begin{frame}{Example of Two-Step fit:}
	Fit the parameters with order $3$ polynomials
	\begin{figure}
        \includegraphics[width=0.95\textwidth]{fit1DS4}
    \end{figure}
\end{frame}

\begin{frame}{Example of Two-Step fit:}
	Use the obtained parameters to define a $2$D function
	\begin{figure}
        \includegraphics[width=0.95\textwidth]{fit1DS5}
    \end{figure}
\end{frame}

\begin{frame}{Example of Two-Step fit $(1., 1, 3, 3, 3)$}
	Result:
	\begin{figure}
        \includegraphics[width=0.95\textwidth]{formationFit1D_13331}
    \end{figure}
\end{frame}

\begin{frame}{Selection Criteria}
	\large
    \begin{itemize}
      	\item[$\blacktriangleright$] The closest to the formation energies.
      	\newline
      	\item[$\blacktriangleright$] Polynomial orders lower or equal to $3$.
    \end{itemize}
\end{frame}

\begin{frame}{Best $2$D fit $(1.1, 3, 3)$}
	\begin{figure}
        \includegraphics[width=0.95\textwidth]{formationFit2D_3311}
    \end{figure}
\end{frame}

\begin{frame}{Best Two-Step fit $(1., 2, 3, 3, 3)$}
	\begin{figure}
        \includegraphics[width=0.95\textwidth]{formationFit1D_23331}
    \end{figure}
\end{frame}

\begin{frame}{Binding Energies, $2$D fit $(1.1, 3, 3)$}
	\begin{figure}
        \includegraphics[width=0.95\textwidth]{bindingFit2D_3311}
    \end{figure}
\end{frame}

\begin{frame}{Binding Energies, Two-Step fit $(1., 2, 3, 3, 3)$}
	\begin{figure}
        \includegraphics[width=0.95\textwidth]{bindingFit1D_23331}
    \end{figure}
\end{frame}

\begin{frame}{Confidence Interval of the Fit}
	\large
	Compare the given formation energies to the fitted ones:
    \begin{itemize}
      	\item[$\blacktriangleright$] compute the distance between them for each
      	vacancy number
      	\item[$\blacktriangleright$] plot it in a histogram and fit it with a
      	normal distribution \newline
    \end{itemize}
    \normalsize
    Example for V~$=32$: 
	\begin{figure}
        \includegraphics[width=0.6\textwidth]{confidenceInterval32}
    \end{figure}
\end{frame}

\begin{frame}{Confidence Interval of the Fit}
	Summary for all vacancy numbers:
	\begin{figure}
        \includegraphics[width=0.8\textwidth]{CIFit1D_23331}
    \end{figure}
\end{frame}

\begin{frame}{Confidence Interval of the Fit}
	\begin{figure}
        \includegraphics[width=0.9\textwidth]{CIComparison}
    \end{figure}
\end{frame}

\begin{frame}{Problem}
	\Large
	Most of the fit tried to give a non-decreasing function of He for V~$= 1$.
\end{frame}

\begin{frame}{Fitting binding energies for V~$= 1,2$}
     \vspace{3mm}
     \large
     Two options\ldots
     \begin{itemize}
           \item[$\blacktriangleright$] Due to the lack of data, use fit 
			described
           in previous slides.
           \item[$\blacktriangleright$] Acquire a good fit for V~$= 1,2$ 
			ONLY and
           use previously described fit for all V~$> 2$ \newline
     \end{itemize}
     \normalsize
     \vspace{-1.5mm}
     \begin{figure}
         \includegraphics[width=0.6\linewidth]{V1_2_fit}
     \end{figure}
\end{frame}

\begin{frame}{Bayesian Inference Principle}
	\Large
	$$P(H|E) \propto P(E|H) \cdot P(H)$$ \newline

	\large
	\begin{itemize}
  		\item[$\blacktriangleright$] the \textbf{posterior} $P(H|E)$ (probability of
  		the hypothesis $H$ given the evidence $E$) that is infered
  		\item[$\blacktriangleright$] the \textbf{likelihood} $P(E|H)$ (probability
  		of the evidence $E$ given the hypothesis $H$)
  		\item[$\blacktriangleright$] and the \textbf{prior} $P(H)$ that gathers all
  		the information one had before the evidence $E$ was observed
	\end{itemize}
	
\end{frame}

\begin{frame}{Bayesian Inference in UQTk}
	\large
	Markov Chain Monte Carlo (MCMC) Method: Metropolis-Hastings algorithm with
	adaptive proposal distribution. \newline
    \begin{itemize}
    	\item[$\blacktriangleright$] Metropolis-Hastings: the step $Y$ at $t$ is
    	kept with a probability $\alpha$
    	
    	$$\alpha = \text{min}(1, \frac{\pi (Y)}{\pi (X_{t-1})})$$
    	with $\pi$ the target distribution. \newline
    	
    	\item[$\blacktriangleright$] Adaptive part: the proposal distribution (to go from a step
    to the next one) is a function of all the previous step.
    \end{itemize}
\end{frame}

\begin{frame}{Bayesian Inference in UQTk: Example}
	\large
    \begin{itemize}
    	\item[$\blacktriangleright$] Generate points from $-20$ to $20$ following
    	$$f(x) = -1 + 0.4 x - 0.12 x^2 + 0.032 x^3$$
    	
    	with a gaussian noise of amplitude $5$. \newline
    	
    	\item[$\blacktriangleright$] Model them with
    	$$M(x) = \texttt{param\_a} + \texttt{param\_b} \cdot x - \texttt{param\_c}
    	\cdot x^2 + \texttt{param\_d} \cdot x^3$$
    	
    	and the same gaussian noise.
    \end{itemize}
\end{frame}

\begin{frame}{Bayesian Inference in UQTk: Example}
     \begin{figure}
         \includegraphics[width=0.8\textwidth]{result}
     \end{figure}
\end{frame}

\begin{frame}{Bayesian Inference in UQTk: Example}
     \begin{figure}
         \includegraphics[width=0.8\textwidth]{chainExample}
     \end{figure}
\end{frame}

\begin{frame}{Bayesian Inference in UQTk: Example}
     \begin{figure}
         \includegraphics[width=0.8\textwidth]{chain2DExample}
     \end{figure}
\end{frame}

\begin{frame}{Bayesian Inference in UQTk: Example}
     \begin{figure}
         \includegraphics[width=0.6\textwidth]{posteriorSummary}
     \end{figure}
\end{frame}

\begin{frame}{Bayesian Inference in UQTk: Burn-in}
	\large
	Steps needed for the chain to converge, must not be used to obtain the target
	distribution.
     \begin{figure}
         \includegraphics[width=0.7\textwidth]{burninExample}
     \end{figure}
\end{frame}

\begin{frame}{Bayesian Inference in UQTk: $\gamma$ parameter}
	\large
	``Size of the step", needs to be fine tuned:
    \begin{itemize}
    	\item[$\blacktriangleright$] if too small (high acceptance rate)
     	\begin{figure}
         	\includegraphics[width=0.6\textwidth]{smallGammaExample}
     	\end{figure}
    	
    	\item[$\blacktriangleright$] if too big: low acceptance rate.
    \end{itemize}
\end{frame}

\begin{frame}{Bayesian Inference in UQTk: $V = 27$ Formation Energies}
	\large
	    \begin{itemize}
    	\item[$\blacktriangleright$] Suppose the following function is used to fit
    	the data 
    	$$f(x) = a + b x + c x^2 + d x^3 + g x^4 + h x^5$$
    	\item[$\blacktriangleright$] Fitting the data using GNUplot gives
    	\begin{align*}
    	a = 49.0722, &b = -1.0619, c = -4.87474, d = 9.92562, \\
    	&g = -2.47767, h = 0.200702
    	\end{align*}
    	\item[$\blacktriangleright$] Model the data with
    	\begin{align*}
    	M(x) = \texttt{param\_a} &+ \texttt{param\_b} \cdot x - \texttt{param\_c}
    	\cdot x^2 + 9.92562 \cdot x^3 \\
    	 &- 2.47467 \cdot x^4 + 0.200702 \cdot x^5
    	\end{align*}
    	to infer 3 parameters.
    \end{itemize}
\end{frame}

\begin{frame}{Bayesian Inference in UQTk: V = 27}
     \begin{figure}
         \includegraphics[width=0.95\textwidth]{V27_postpred}
     \end{figure}
\end{frame}

\begin{frame}{Bayesian Inference in UQTk: V = 27}
	\begin{columns}[c]		
		\column{.56\textwidth}
			\begin{figure}
	 	 		\includegraphics[width=\textwidth]{V27_posteriors}
	 		\end{figure}
	 	\column{.43\textwidth}
	 		\center{$\gamma = 0.1$}
	 		\begin{figure}[ht]
	 			\subfigure{\includegraphics[width=\textwidth]{V27_chn_param_a_param_c}}
	 			\newline
	 			\newline
         		\subfigure{\includegraphics[width=\textwidth]{V27_chn_param_c}}
     		\end{figure}	 		
	\end{columns}
\end{frame}

\begin{frame}{Bayesian Inference in UQTk: $V = 27$ Formation Energies}
	\large
	    \begin{itemize}
    	\item[$\blacktriangleright$] Suppose the following function is used to fit
    	the data 
    	$$f(x) = a + b x + c x^2 + d x^3 + g x^4 + h x^5$$
    	\item[$\blacktriangleright$] Fitting the data using GNUplot gives
    	\begin{align*}
    	a = 49.0722, &b = -1.0619, c = -4.87474, d = 9.92562, \\
    	&g = -2.47767, h = 0.200702
    	\end{align*}
    	\item[$\blacktriangleright$] Model the data with
    	\begin{align*}
    	M(x) = \texttt{param\_a} &+ \texttt{param\_b} \cdot x - \texttt{param\_c}
    	\cdot x^2 + \texttt{param\_d} \cdot x^3 \\
    	 &- 2.47467 \cdot x^4 + 0.200702 \cdot x^5
    	\end{align*}
    	to infer 4 parameters.
    \end{itemize}
\end{frame}

\begin{frame}{Bayesian Inference in UQTk: V = 27}
     \begin{figure}
         \includegraphics[width=0.95\textwidth]{V27_postpredabcd}
     \end{figure}
\end{frame}

\begin{frame}{Bayesian Inference in UQTk: V = 27}
	\begin{columns}[c]
		\column{.6\textwidth}
			\begin{figure}
	 	 		\includegraphics[width=\textwidth]{V27abcd_posteriors}
	 		\end{figure} 	
	 	\column{.39\textwidth}
	 		\center{$\gamma = 0.1$}
	 		\begin{figure}[ht]
	 			\subfigure{\includegraphics[width=\textwidth]{V27abcd_chnparamb_d}}
	 			\newline
	 			\newline
         		\subfigure{\includegraphics[width=\textwidth]{V27_chnparama_d}}
     		\end{figure}   		
	\end{columns}
\end{frame}

\begin{frame}{Bayesian Inference in UQTk: V = 27}
	\begin{columns}[c]
	 	\column{.5\textwidth}
	 		\begin{figure}[ht]
	 			\subfigure{\includegraphics[width=\textwidth]{V27abcd_chnparama}}
	 			\newline
         		\subfigure{\includegraphics[width=\textwidth]{V27abcd_chnparamc}}
     		\end{figure}
		\column{.5\textwidth}
	 		\begin{figure}[ht]
	 			\subfigure{\includegraphics[width=\textwidth]{V27abcd_chnparam_b}}
	 			\newline
         		\subfigure{\includegraphics[width=\textwidth]{V27abcd_chnparam_d}}
     		\end{figure}		  		
	\end{columns}
\end{frame}

\begin{frame}{Bayesian Inference in UQTk: V = 27, Infer 5 params}
     \begin{figure}
         \includegraphics[width=0.95\textwidth]{V27most_postpred}
     \end{figure}
\end{frame}

\begin{frame}{Bayesian Inference in UQTk: V = 27, infer 5 params}
	\begin{columns}[c]
		\column{.6\textwidth}
			\begin{figure}
	 	 		\includegraphics[width=\textwidth]{V27most_posteriors}
	 		\end{figure} 	
	 	\column{.39\textwidth}
	 		\center{$\gamma = 0.1$}
	 		\begin{figure}[ht]
	 			\subfigure{\includegraphics[width=\textwidth]{V27most_chnparama_d}}
	 			\newline
	 			\newline
         		\subfigure{\includegraphics[width=\textwidth]{V27most_chnparama}}
     		\end{figure}   		
	\end{columns}
\end{frame}

\begin{frame}{Bayesian Inference on Formation Energies}
	\large
	\begin{itemize}
	  \item[$\blacktriangleright$] Using Bayesian inference to infer the parameters
	  of the formation energy fit has proven to be a nontrivial task \newline
	  \item[$\blacktriangleright$] After many unsuccessful attempts to infer
	  the fit parameters for just one vacancy number further investigation into
	  the formation energy noise was performed
	\end{itemize}
\end{frame}

\begin{frame}{Formation Energy Noise for V = 1, 2, 6, 14}
     \begin{figure}
         \includegraphics[width=0.43\textwidth]{V1noise}
         \hspace{4mm}
         \includegraphics[width=0.43\textwidth]{V2noise}
     \end{figure}
     \vspace{-2mm}
     \begin{figure}
         \includegraphics[width=0.43\textwidth]{V6noise}
         \hspace{4mm}
         \includegraphics[width=0.43\textwidth]{V14noise}
     \end{figure}
\end{frame}

\begin{frame}{Formation Energy Noise for V = 18, 19, 27, 32, 44}
     \vspace{-2mm}
     \begin{figure}[!htp]
         \includegraphics[width=0.33\textwidth]{V18noise}
         \hspace{4mm}
         \includegraphics[width=0.33\textwidth]{V19noise}
     \end{figure}
     \vspace{-3mm}
     \begin{figure}[!htp]
         \includegraphics[width=0.33\textwidth]{V27noise}
         \includegraphics[width=0.33\textwidth]{V32noise}
         \includegraphics[width=0.33\textwidth]{V44noise}
     \end{figure}
\end{frame}

\begin{frame}{Fit vs. Bayesian Noise for V = 27}
	\begin{figure}
		\includegraphics[width=0.5\linewidth]{V27fitnoise}
	\end{figure}
	\vspace{-4mm}
	\begin{figure}
		\includegraphics[width=0.5\linewidth]{V27_bayesian_noise}
	\end{figure}
\end{frame}


\begin{frame}{Bayesian Inference in UQTk: V = 18, infer 6 params}
	\begin{columns}[c]
		\column{.6\textwidth}
			\begin{figure}
	 	 		\includegraphics[width=\textwidth]{v18chain}
	 		\end{figure} 	
	 	\column{.39\textwidth}
	 		\begin{figure}[ht]
	 			\subfigure{\includegraphics[width=\textwidth]{v18_result6params}}
	 			\newline
	 			\newline
         		\subfigure{\includegraphics[width=\textwidth]{v18_noise6params}}
     		\end{figure}   		
	\end{columns}
\end{frame}

\begin{frame}{Bayesian Inference in UQTk: V = 32, infer 6 params}
	\begin{columns}[c]
		\column{.6\textwidth}
			\begin{figure}
	 	 		\includegraphics[width=\textwidth]{v32chain}
	 		\end{figure} 	
	 	\column{.39\textwidth}
	 		\begin{figure}[ht]
	 			\subfigure{\includegraphics[width=\textwidth]{v32_6params}}
	 			\newline
	 			\newline
         		\subfigure{\includegraphics[width=\textwidth]{v32noise6params}}
     		\end{figure}   		
	\end{columns}
\end{frame}

\begin{frame}{Piecewise Bayesian Inference V~$= 1$}
  	\begin{columns}[onlytextwith]
    	\begin{column}{0.43\textwidth}
    	$$He/V \leq 1$$
    	\vspace{1.2cm}
      		\begin{figure}
        		\includegraphics[width=0.9\textwidth]{low1Triangle}
      		\end{figure}
    	\end{column}  
    	\begin{column}{0.59\textwidth}
    	$$He/V \geq 1$$
      		\begin{figure}
        		\includegraphics[width=0.9\textwidth]{high1Triangle}
      		\end{figure}
    	\end{column}
  	\end{columns}
\end{frame}

\begin{frame}{Piecewise Bayesian Inference V~$= 1$}
  	\begin{columns}[onlytextwith]
    	\begin{column}{0.5\textwidth}
    	$$He/V \leq 1$$
      		\begin{figure}
        		\includegraphics[width=0.9\textwidth]{low1Result}
      		\end{figure}
      		~~~~~~~~~~~~~~~~~~~$2$~points
    	\end{column}  
    	\begin{column}{0.5\textwidth}
    	$$He/V \geq 1$$
      		\begin{figure}
        		\includegraphics[width=0.9\textwidth]{high1Result}
      		\end{figure}
      		~~~~~~~~~~~~~~~~~~~$9$~points
    	\end{column}
  	\end{columns}
\end{frame}

\begin{frame}{Piecewise Bayesian Inference V~$= 1$}
  	\begin{columns}[onlytextwith]
    	\begin{column}{0.5\textwidth}
    	$$He/V \leq 1$$
      		\begin{figure}
        		\includegraphics[width=0.9\textwidth]{low1ResultDiff}
      		\end{figure}
      		~~~~~~~~~~~~~~~~~~~$2$~points
    	\end{column}  
    	\begin{column}{0.5\textwidth}
    	$$He/V \geq 1$$
      		\begin{figure}
        		\includegraphics[width=0.9\textwidth]{high1ResultDiff}
      		\end{figure}
      		~~~~~~~~~~~~~~~~~~~$9$~points
    	\end{column}
  	\end{columns}
\end{frame}

\begin{frame}{Piecewise Bayesian Inference V~$= 44$}
  	\begin{columns}[onlytextwith]
    	\begin{column}{0.43\textwidth}
    	$$He/V \leq 1$$
    	\vspace{1.2cm}
      		\begin{figure}
        		\includegraphics[width=0.9\textwidth]{low44Triangle}
      		\end{figure}
    	\end{column}  
    	\begin{column}{0.59\textwidth}
    	$$He/V \geq 1$$
      		\begin{figure}
        		\includegraphics[width=0.9\textwidth]{high44Triangle}
      		\end{figure}
    	\end{column}
  	\end{columns}
\end{frame}

\begin{frame}{Piecewise Bayesian Inference V~$= 44$}
  	\begin{columns}[onlytextwith]
    	\begin{column}{0.5\textwidth}
    	$$He/V \leq 1$$
      		\begin{figure}
        		\includegraphics[width=0.9\textwidth]{low44Result}
      		\end{figure}
      		~~~~~~~~~~~~~~~~~~~$45$~points
    	\end{column}  
    	\begin{column}{0.5\textwidth}
    	$$He/V \geq 1$$
      		\begin{figure}
        		\includegraphics[width=0.9\textwidth]{high44Result}
      		\end{figure}
      		~~~~~~~~~~~~~~~~~~~$146$~points
    	\end{column}
  	\end{columns}
\end{frame}

\begin{frame}{Piecewise Bayesian Inference V~$= 44$}
  	\begin{columns}[onlytextwith]
    	\begin{column}{0.5\textwidth}
    	$$He/V \leq 1$$
      		\begin{figure}
        		\includegraphics[width=0.9\textwidth]{low44ResultDiff}
      		\end{figure}
      		~~~~~~~~~~~~~~~~~~~$45$~points
    	\end{column}  
    	\begin{column}{0.5\textwidth}
    	$$He/V \geq 1$$
      		\begin{figure}
        		\includegraphics[width=0.9\textwidth]{high44ResultDiff}
      		\end{figure}
      		~~~~~~~~~~~~~~~~~~~$146$~points
    	\end{column}
  	\end{columns}
\end{frame}

\end{document}